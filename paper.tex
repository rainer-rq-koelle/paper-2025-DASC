% *** Authors should verify (and, if needed, correct) their LaTeX system  ***
% *** with the testflow diagnostic prior to trusting their LaTeX platform ***
% *** with production work. IEEE's font choices can trigger bugs that do  ***
% *** not appear when using other class files.                            ***
% The testflow support page is at:
% http://www.michaelshell.org/tex/testflow/


%%*************************************************************************
%% Legal Notice:
%% This code is offered as-is without any warranty either expressed or
%% implied; without even the implied warranty of MERCHANTABILITY or
%% FITNESS FOR A PARTICULAR PURPOSE!
%% User assumes all risk.
%% In no event shall IEEE or any contributor to this code be liable for
%% any damages or losses, including, but not limited to, incidental,
%% consequential, or any other damages, resulting from the use or misuse
%% of any information contained here.
%%
%% All comments are the opinions of their respective authors and are not
%% necessarily endorsed by the IEEE.
%%
%% This work is distributed under the LaTeX Project Public License (LPPL)
%% ( http://www.latex-project.org/ ) version 1.3, and may be freely used,
%% distributed and modified. A copy of the LPPL, version 1.3, is included
%% in the base LaTeX documentation of all distributions of LaTeX released
%% 2003/12/01 or later.
%% Retain all contribution notices and credits.
%% ** Modified files should be clearly indicated as such, including  **
%% ** renaming them and changing author support contact information. **
%%
%% File list of work: IEEEtran.cls, New_IEEEtran_how-to.pdf, bare_jrnl_new_sample4.tex,
%%*************************************************************************
\PassOptionsToPackage{unicode}{hyperref}
\PassOptionsToPackage{hyphens}{url}
\PassOptionsToPackage{dvipsnames,svgnames,x11names}{xcolor}
% Note that the a4paper option is mainly intended so that authors in
% countries using A4 can easily print to A4 and see how their papers will
% look in print - the typesetting of the document will not typically be
% affected with changes in paper size (but the bottom and side margins will).
% Use the testflow package mentioned above to verify correct handling of
% both paper sizes by the user's LaTeX system.
%
% Also note that the "draftcls" or "draftclsnofoot", not "draft", option
% should be used if it is desired that the figures are to be displayed in
% draft mode.
%
\documentclass[
  journal,
]{IEEEtran}%
% If IEEEtran.cls has not been installed into the LaTeX system files,
% manually specify the path to it like:
% \documentclass[journal]{../sty/IEEEtran}
\usepackage[cmex10]{amsmath}
\usepackage{amssymb}
\usepackage{iftex}
\ifPDFTeX
  \usepackage[T1]{fontenc}
  \usepackage[utf8]{inputenc}
  \usepackage{textcomp} % provide euro and other symbols
\else % if luatex or xetex
  \usepackage{unicode-math} % this also loads fontspec
  \defaultfontfeatures{Scale=MatchLowercase}
  \defaultfontfeatures[\rmfamily]{Ligatures=TeX,Scale=1}
\fi
%\usepackage{lmodern}
\ifPDFTeX\else
\fi
% Use upquote if available, for straight quotes in verbatim environments
\IfFileExists{upquote.sty}{\usepackage{upquote}}{}
\IfFileExists{microtype.sty}{% use microtype if available
  \usepackage[]{microtype}
  \UseMicrotypeSet[protrusion]{basicmath} % disable protrusion for tt fonts
}{}
\makeatletter
\parindent    1.0em
\ifCLASSOPTIONcompsoc
  \parindent    1.5em
\fi
\makeatother
\usepackage{xcolor}
\setlength{\emergencystretch}{3em} % prevent overfull lines

\setcounter{secnumdepth}{5}
% Make \paragraph and \subparagraph free-standing
\ifx\paragraph\undefined\else
  \let\oldparagraph\paragraph
  \renewcommand{\paragraph}[1]{\oldparagraph{#1}\mbox{}}
\fi
\ifx\subparagraph\undefined\else
  \let\oldsubparagraph\subparagraph
  \renewcommand{\subparagraph}[1]{\oldsubparagraph{#1}\mbox{}}
\fi


\providecommand{\tightlist}{%
  \setlength{\itemsep}{0pt}\setlength{\parskip}{0pt}}\usepackage{longtable,booktabs,array}
\usepackage{calc} % for calculating minipage widths
% Correct order of tables after \paragraph or \subparagraph
\usepackage{etoolbox}
\makeatletter
\patchcmd\longtable{\par}{\if@noskipsec\mbox{}\fi\par}{}{}
\makeatother
% Allow footnotes in longtable head/foot
\IfFileExists{footnotehyper.sty}{\usepackage{footnotehyper}}{\usepackage{footnote}}
\makesavenoteenv{longtable}
\usepackage{graphicx}
\makeatletter
\def\maxwidth{\ifdim\Gin@nat@width>\linewidth\linewidth\else\Gin@nat@width\fi}
\def\maxheight{\ifdim\Gin@nat@height>\textheight\textheight\else\Gin@nat@height\fi}
\makeatother
% Scale images if necessary, so that they will not overflow the page
% margins by default, and it is still possible to overwrite the defaults
% using explicit options in \includegraphics[width, height, ...]{}
\setkeys{Gin}{width=\maxwidth,height=\maxheight,keepaspectratio}
% Set default figure placement to htbp
\makeatletter
\def\fps@figure{htbp}
\makeatother
% definitions for citeproc citations
\NewDocumentCommand\citeproctext{}{}
\NewDocumentCommand\citeproc{mm}{%
  \begingroup\def\citeproctext{#2}\cite{#1}\endgroup}
\makeatletter
 % allow citations to break across lines
 \let\@cite@ofmt\@firstofone
 % avoid brackets around text for \cite:
 \def\@biblabel#1{}
 \def\@cite#1#2{{#1\if@tempswa , #2\fi}}
\makeatother
\newlength{\cslhangindent}
\setlength{\cslhangindent}{1.5em}
\newlength{\csllabelwidth}
\setlength{\csllabelwidth}{3em}
\newenvironment{CSLReferences}[2] % #1 hanging-indent, #2 entry-spacing
 {\begin{list}{}{%
  \setlength{\itemindent}{0pt}
  \setlength{\leftmargin}{0pt}
  \setlength{\parsep}{0pt}
  % turn on hanging indent if param 1 is 1
  \ifodd #1
   \setlength{\leftmargin}{\cslhangindent}
   \setlength{\itemindent}{-1\cslhangindent}
  \fi
  % set entry spacing
  \setlength{\itemsep}{#2\baselineskip}}}
 {\end{list}}
\usepackage{calc}
\newcommand{\CSLBlock}[1]{\hfill\break\parbox[t]{\linewidth}{\strut\ignorespaces#1\strut}}
\newcommand{\CSLLeftMargin}[1]{\parbox[t]{\csllabelwidth}{\strut#1\strut}}
\newcommand{\CSLRightInline}[1]{\parbox[t]{\linewidth - \csllabelwidth}{\strut#1\strut}}
\newcommand{\CSLIndent}[1]{\hspace{\cslhangindent}#1}

\usepackage{physics}
\usepackage[version=3]{mhchem}
\usepackage{orcidlink}
\usepackage{float}
\floatplacement{table}{htb}
\makeatletter
\@ifpackageloaded{tcolorbox}{}{\usepackage[skins,breakable]{tcolorbox}}
\@ifpackageloaded{fontawesome5}{}{\usepackage{fontawesome5}}
\definecolor{quarto-callout-color}{HTML}{909090}
\definecolor{quarto-callout-note-color}{HTML}{0758E5}
\definecolor{quarto-callout-important-color}{HTML}{CC1914}
\definecolor{quarto-callout-warning-color}{HTML}{EB9113}
\definecolor{quarto-callout-tip-color}{HTML}{00A047}
\definecolor{quarto-callout-caution-color}{HTML}{FC5300}
\definecolor{quarto-callout-color-frame}{HTML}{acacac}
\definecolor{quarto-callout-note-color-frame}{HTML}{4582ec}
\definecolor{quarto-callout-important-color-frame}{HTML}{d9534f}
\definecolor{quarto-callout-warning-color-frame}{HTML}{f0ad4e}
\definecolor{quarto-callout-tip-color-frame}{HTML}{02b875}
\definecolor{quarto-callout-caution-color-frame}{HTML}{fd7e14}
\makeatother
\makeatletter
\@ifpackageloaded{caption}{}{\usepackage{caption}}
\AtBeginDocument{%
\ifdefined\contentsname
  \renewcommand*\contentsname{Table of contents}
\else
  \newcommand\contentsname{Table of contents}
\fi
\ifdefined\listfigurename
  \renewcommand*\listfigurename{List of Figures}
\else
  \newcommand\listfigurename{List of Figures}
\fi
\ifdefined\listtablename
  \renewcommand*\listtablename{List of Tables}
\else
  \newcommand\listtablename{List of Tables}
\fi
\ifdefined\figurename
  \renewcommand*\figurename{Fig.}
\else
  \newcommand\figurename{Fig.}
\fi
\ifdefined\tablename
  \renewcommand*\tablename{Table}
\else
  \newcommand\tablename{Table}
\fi
}
\@ifpackageloaded{float}{}{\usepackage{float}}
\floatstyle{ruled}
\@ifundefined{c@chapter}{\newfloat{codelisting}{h}{lop}}{\newfloat{codelisting}{h}{lop}[chapter]}
\floatname{codelisting}{Listing}
\newcommand*\listoflistings{\listof{codelisting}{List of Listings}}
\makeatother
\makeatletter
\makeatother
\makeatletter
\@ifpackageloaded{caption}{}{\usepackage{caption}}
\@ifpackageloaded{subcaption}{}{\usepackage{subcaption}}
\makeatother
\usepackage[skip=2pt,font=footnotesize]{caption}
%\captionsetup{format=myformat}
\makeatletter
%\setlength{\cslhangindent}{0pt plus .5pt}
\providecommand{\bibfont}{\footnotesize}
\let\CSLReferences@rig=\CSLReferences
\renewcommand{\CSLReferences}[2]{
\bibfont\settowidth\csllabelwidth{[999]}
\CSLReferences@rig{#1}{#2}
\vskip 0.3\baselineskip plus 0.1\baselineskip minus 0.1\baselineskip%
}
\makeatother
\ifLuaTeX
  \usepackage{selnolig}  % disable illegal ligatures
\fi
\IfFileExists{bookmark.sty}{\usepackage{bookmark}}{\usepackage{hyperref}}
\IfFileExists{xurl.sty}{\usepackage{xurl}}{} % add URL line breaks if available
\urlstyle{same} % disable monospaced font for URLs
\hypersetup{
  pdftitle={Towards Measuring Operational Efficiency of Arrival Management Techniques},
  colorlinks=true,
  linkcolor={blue},
  filecolor={Maroon},
  citecolor={Blue},
  urlcolor={Blue},
  pdfcreator={LaTeX via pandoc}}

% *** Do not adjust lengths that control margins, column widths, etc. ***
% *** Do not use packages that alter fonts (such as pslatex).         ***
% There should be no need to do such things with IEEEtran.cls V1.6 and later.
% (Unless specifically asked to do so by the journal or conference you plan
% to submit to, of course. )


% correct bad hyphenation here
\hyphenation{op-tical net-works semi-conduc-tor}

%
% paper title
% can use linebreaks \\ within to get better formatting as desired
% Do not put math or special symbols in the title.
% paper title
% can use linebreaks \\ within to get better formatting as desired
% Do not put math or special symbols in the title.
\title{Towards Measuring Operational Efficiency of Arrival Management
Techniques}

\author{

}
\begin{document}

% The paper headers

% use for special paper notices

% make the title area
\maketitle

% As a general rule, do not put math, special symbols or citations
% in the abstract or keywords.
\begin{abstract}
Flight efficiency in the arrival phase offers a significant benefit pool
to influence fuel burn and associated emissions. Today no operational
performance metric exists that allows to assess the effectiveness of
different arrival management techniques. This paper characterises and
compares arrival management at a series of airports and develops a
proposed operational performance measure combining vertical and
horizontal flight efficiency, including air time and associated fuel
burn and emissions. While the approach focusses on the arrival phase,
the concept could be expanded to assess the operational efficiency of
trajectory based operations.
\end{abstract}
% Note that keywords are not normally used for peerreview papers.

% For peer review papers, you can put extra information on the cover
% page as needed:
% \ifCLASSOPTIONpeerreview
% \begin{center} \bfseries EDICS Category: 3-BBND \end{center}
% \fi
%
% For peerreview papers, this IEEEtran command inserts a page break and
% creates the second title. It will be ignored for other modes.
% \IEEEpeerreviewmaketitle


\section{Introduction}\label{introduction}

Political priorities have focussed on the climate impact of air
transportation and culminated in ICAO's long-term aspirational goal to
substantially reduce CO2 emissions. From a temporal perspective, the
introduction of new aircraft airframe design or novel propulsion
technology will require substantial developments and are only a
long-term measure. The uptake of sustainable aviation fuel is still in
its infancy. Accordingly, increased operational efficiency may come from
utilising existing benefit pools in terms of air traffic management.
Such benefits can be exploited immediately.

Arrival operations at airports contribute to managing planning
uncertainty, actual travel times, and limited runway system throughput
capacity. This paper considers a data-driven approach to characterise
and monitor the operational performance of arrival management in Europe,
the United States, and Brazil. The overall goal is to establish a
performance measure that will support the assessment of constraints and
the exploitation of the anticipated benefit pools by combining the
classical horizontal, vertical, and temporal approaches from a
trajectory perspective.

Trajectory-based operations are a means to balance the utilisation of
the runway system capacity and associated arrival traffic
synchronisation. The success of avoiding air traffic control
interventions within the terminal airspace diminishes without utilising
upstream synchronisation of arrival traffic. However, upstream
synchronisation activities may be subject to a variety of impact such as
departing traffic or crossing and arrivals to adjacent airports.

Understanding constraints on the traffic synchronisation before entering
the terminal airspace will help to improve airspace and procedure
design. This paper builds on a spatial-temporal framework for arrival
management by analysing flight trajectories and aims at the
identification of potential sources of inefficiencies in the wider
arrival airspace. The trajectory-based approach overcomes the
limitations of today's operational performance monitoring metrics and
allows for the inclusion of meteorological or airspace structure related
constraints, and support to evaluate the operated sequencing
techniques/concepts.

This paper addresses the challenges of characterising and monitoring the
performance of arrival management techniques. The approach is detailed
and studied on the basis of open trajectory data for a subset of
airports with different arrival concepts (e.g.~runway
pressure/time-based separation on final, point-merge, extended arrival
management) in Europe, the United States, and Brazil. The modelling work
includes a framework to characterise arrival management constraints. The
validation of the approach has been tested based on historic open
trajectory data. The results obtained indicate the general feasibility
of the approach and its application within the different operational
contexts. The operational analysis support a richer discussion of
efficiency increasing improvements and concepts on the local level. The
ability to provide spatial and temporal measures based on 4D
trajectories is an enabler to assess and compare the benefits of
specific arrival management techniques, address airspace design, and
operational concepts. This allows strategic planers and decision-makers
to select from a wider range of such techniques and options.

Higher levels of operational efficiency will support the quest to reduce
overall emissions of air transportation and meet the environmental
goals. The work presented serves as a building block to develop a
data-driven approach to measuring performance in the arrival phase. It
also reflects the development of a framework to discuss the utility of
upstream arrival management techniques at different airports and
required operational changes. This also offers to extent the approach
and generalise it for the full trajectory.

\section{Background}\label{background}

\subsection{Arrival Management}\label{arrival-management}

\begin{itemize}
\tightlist
\item
  what is arival management - how do we define it
\item
  what is done by others
\item
  techniques used (e.g.~runway pressure/time-based separation on final,
  point-merge, extended arrival management) in Europe, the United
  States, and Brazil.
\item
  list some examples in the different places.
\end{itemize}

Arrival management is a well researched field. Various approaches exist
ranging from statistical procedures to simulations. For example, Itoh et
al \citeproc{ref-eri2020arrivalprocess}{{[}1{]}} investigates flow
characteristics for arrivals (SAY SOMETHING MEANINGFUL).

A series of papers studied \emph{point merge operations} (e.g.~CITE SOME
\citeproc{ref-christien_2017}{{[}2{]}}). Our approach builds on the
concept of \emph{spacing deviation} introduced by
\citeproc{ref-christien_2017}{{[}2{]}}.

\subsection{Global Air Navigation Plan - Operational
Performance}\label{global-air-navigation-plan---operational-performance}

ICAO's Global Air Navigation Plan (GANP) proposes to measure operational
performance with a set of key performance indicators (KPIs). The latter
offer a measurement for a specific flight phase. There exists criticism
that the ICAO KPIs do not address the gate-to-gate perspective. In
particular, these indicators appear to be not suitable to measure
trajectory-based operations (TBO).

\section{Conceptual Approach, Data, and
Methods}\label{conceptual-approach-data-and-methods}

\subsection{Data}\label{data}

\begin{itemize}
\tightlist
\item
  Open data = x months of data for airport 1 \ldots{} n, 205NM
\end{itemize}

\section{Results and Discussion}\label{results-and-discussion}

\subsection{Characterising Operations with GANP
KPIs}\label{characterising-operations-with-ganp-kpis}

\begin{itemize}
\tightlist
\item
  calculate KPIs for study airports

  \begin{itemize}
  \tightlist
  \item
    runway throughput (arrival throughput)
  \item
    ASMA (40NM, 100NM, 200NM)
  \item
    vertical flight efficiency (200NM)
  \end{itemize}
\item
  what can we say about these airports and (measured) operations?
\item
  abstract speaks about meteo - can we use METAR to do something here?
  Possibly say something about wind-situation
\end{itemize}

\subsection{Measuring Spacing
Deviation}\label{measuring-spacing-deviation}

\begin{tcolorbox}[enhanced jigsaw, titlerule=0mm, coltitle=black, colback=white, breakable, toprule=.15mm, arc=.35mm, colbacktitle=quarto-callout-note-color!10!white, bottomtitle=1mm, opacitybacktitle=0.6, opacityback=0, leftrule=.75mm, left=2mm, toptitle=1mm, title=\textcolor{quarto-callout-note-color}{\faInfo}\hspace{0.5em}{Note}, colframe=quarto-callout-note-color-frame, rightrule=.15mm, bottomrule=.15mm]

\begin{itemize}
\tightlist
\item
  come up with a measurement of the realised spacing deviation
\end{itemize}

\end{tcolorbox}

\section{Conclusions}\label{conclusions}

The results of this study will be shared with the wider ICAO Performance
Community for a potential inclusion in the next iteration of the Global
Air Navigation Plan.

\section*{References}\label{references}
\addcontentsline{toc}{section}{References}

\phantomsection\label{refs}
\begin{CSLReferences}{0}{0}
\bibitem[\citeproctext]{ref-eri2020arrivalprocess}
\CSLLeftMargin{{[}1{]} }%
\CSLRightInline{I. Eri, M. Schultz, A. Srinivas, and V. Duong,
{``Devising strategies for aircraft arrival processes via distance-based
queuing models,''} 2020 {[}Online{]}. Available:
\url{https://www.sesarju.eu/sites/default/files/documents/sid/2020/papers/SIDs_2020_paper_24red.pdf}}

\bibitem[\citeproctext]{ref-christien_2017}
\CSLLeftMargin{{[}2{]} }%
\CSLRightInline{R. Christien, E. Hoffman, A. Trzmiel, and K. Zeghal,
{``Toward the characterisation of sequencing arrivals,''} in \emph{12th
{USA}/{Europe} air traffic management {R}\&{D} seminar, seattle, {USA}},
2017. }

\end{CSLReferences}


% Can use something like this to put references on a page
% by themselves when using endfloat and the captionsoff option.
\ifCLASSOPTIONcaptionsoff
  \newpage
\fi

% trigger a \newpage just before the given reference
% number - used to balance the columns on the last page
% adjust value as needed - may need to be readjusted if
% the document is modified later
%\IEEEtriggeratref{8}
% The "triggered" command can be changed if desired:
%\IEEEtriggercmd{\enlargethispage{-5in}}

% Uncomment when use biblatex with style=ieee
%\renewcommand{\bibfont}{\footnotesize} % for IEEE bibfont size

\pagebreak[3]
% that's all folks
\end{document}

